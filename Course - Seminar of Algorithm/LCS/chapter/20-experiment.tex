\chapter{实验仿真}
\section{数据设置}
\subsection{常规数据}
长度为$N$的字符串A由ASCII码$32\sim 125_{(DEC)}$的字符组成,字符串B由A不超过$\textbf{max}(N/100000,20)$次修改
\footnote[1]{单次修改操作为一下操作中的一项:将A中一个字符删除、向任意位置插入一个符合要求的字符或者将任意位置的字符修改}
\subsection{特殊数据}
由于本篇论文尤其对RNA配对的应用进行了研究,所以我们也设置了RNA数据仿真。
此种数据与常规数据的唯一区别在于其中的字符构成仅有A,G,C,U构成
\subsection{数据规模}
字符串A的长度分别取$1000,10000,100000,1000000,5000000,10000000$进行测试,每种长度生成100组数据,每组数据运行10次,取总平均运行时间来表示其运行时间
\subsection{测试环境}
系统:    Windows 10 Pro 1803 17134.112 \par
处理器:  Intel(R) Core(TM) i5-7300HQ CPU @ 2.50GHz 2.50GHz \par
运行内存:8.00 GB \par
编译环境:TDM64-GCC 4.9.2
\section{实验结果}
\subsection{常规数据}
\begin{table}[H]
  \caption{常规数据实验结果}
  \centering
  \begin{tabular}{p{2.6cm}<{\centering}p{1.9cm}<{\centering}p{1.9cm}<{\centering}p{1.9cm}<{\centering}p{1.9cm}<{\centering}p{1.9cm}<{\centering}p{1.9cm}<{\centering}}
  \toprule
   数据规模 & 1000 & 10000 & 100000 & 1000000 & 5000000 & 10000000\\
  \midrule
   原LCS & 7.86ms & 892.16ms & 91.62s & >500s & >500s & >500s\\
   本文的LCS & <0.01ms & 0.27ms & 1.83s & 18.91s & 67.43s & 162.87s\\
  \bottomrule
  \end{tabular}
\end{table}
  
\subsection{特殊数据}
\begin{table}[H]
  \caption{特殊数据实验结果}
  \centering
  \begin{tabular}{p{2.6cm}<{\centering}p{1.9cm}<{\centering}p{1.9cm}<{\centering}p{1.9cm}<{\centering}p{1.9cm}<{\centering}p{1.9cm}<{\centering}p{1.9cm}<{\centering}}
  \toprule
   数据规模 & 1000 & 10000 & 100000 & 1000000 & 5000000 & 10000000\\
  \midrule
   原LCS & 7.18ms & 878.44ms & 79.29s & >500s & >500s & >500s\\
   本文的LCS & <0.01ms & 0.16ms & 1.14s & 8.16s & 61.23s & 128.04s\\
  \bottomrule
  \end{tabular}
\end{table}