\documentclass[
  degree=bachelor,
  type=paper,
  oneside,openany,
  AutoFakeBold=2.5
]{nuaathesis}

\usepackage{algorithm}
\usepackage{algorithmicx}
\usepackage{algpseudocode}
\usepackage{float}
\usepackage{amsmath}
\floatname{algorithm}{算法}
\renewcommand{\algorithmicrequire}{\textbf{输入:}}
\renewcommand{\algorithmicensure}{\textbf{输出:}}
\makeatletter
\newenvironment{breakablealgorithm}
  {% \begin{breakablealgorithm}
   \begin{center}
     \refstepcounter{algorithm}% New algorithm
     \hrule height.8pt depth0pt \kern2pt% \@fs@pre for \@fs@ruled 画线
     \renewcommand{\caption}[2][\relax]{% Make a new \caption
       {\raggedright\textbf{\ALG@name~\thealgorithm} ##2\par}%
       \ifx\relax##1\relax % #1 is \relax
         \addcontentsline{loa}{algorithm}{\protect\numberline{\thealgorithm}##2}%
       \else % #1 is not \relax
         \addcontentsline{loa}{algorithm}{\protect\numberline{\thealgorithm}##1}%
       \fi
       \kern2pt\hrule\kern2pt
     }
  }{% \end{breakablealgorithm}
     \kern2pt\hrule\relax% \@fs@post for \@fs@ruled 画线
   \end{center}
  }
\makeatother
% 写作时,使用这个命令只渲染你想查看的部分,提升工作效率,定稿时注释掉整行
% \includeonly{chapter/00-cover,chapter/30-issues,}

\begin{document}

% 首先,导入论文的基础信息
%# -*- coding: utf-8-unix -*-

% 中文文档信息
\nuaaset{
  title = {faster algorithms on constrained single-source shortest path}, % 论文题目
  author = {李想},   % 作者
  studentid = {161610323},  % 学号(本科)
  college = {计算机科学与技术学院},    % 学院,或者(金城)院部
  major = {计算机软件培优班},     % 专业(本科)
  classid = {1616001},  % 班号(本科)
  advisors = {陈松灿教授},  % 指导教师
  libraryclassid = {TP371},       % 中图分类号(硕博)
  subjectclassid = {080605},      % 学科分类号(硕博)
  thesisid = {1028704 17-S036},   % 论文编号(硕博)
  majorsubject = {学位论文排版},  % 学科、专业(硕博)
  researchfield = {排版引用},     % 研究方向(硕博)
  % applydate = {二〇一七年七月}  % 默认当前日期
}

% 英文文档信息(本科可以不用写)
\nuaasetEn{
  college = {College of Computer Science and Technology},
  title = {Buckets, Heaps, Lists, and Monotone Priority Queues}
  majorsubject = {Computer Advanced Class},
  author = {Lixiang},
  advisors = {Prof.~Chen SongCan},
  % applydate = {July, 2017}
}

%% 中文摘要
\begin{abstract}

\end{abstract}
\keywords{}




\makecover    % 封面
% \makedeclare  % 承诺书

\frontmatter  % 启用页眉页脚
%\makeabstract % 摘要
\nuaatableofcontents  % 正文目录

\mainmatter   % 以下正文

\chapter{引言}
在Dijkstra算法及其扩展专题中,我们小组的研讨方向是的对边权大小种类较少的一类优化,基数堆优化,多重基数堆优化三种优化方法。在对比实验中,我们主要就如何优化算法,优化的目标以及不同优化算法之间的关系进行了讨论与总结。


\chapter{算法简介}

\section{简介}
本文提出了一种名为AB-DCST(Ant-Based algorithm for the DCMST problem),它和其他智能算法一样,循环操作一直到一个边界停止。
每次循环主要包含两个部分,exploration和construction
在exploration部分,蚂蚁们去发现较优的边,把他们放进最小生成树的候选集合
在construction部分中,我们将从候选集合中选出N-1条边建立最小生成树
这样每一次循环都会获得一个可行解,我们就可以从这些可行解中找到极小解作为最终的解。



\section{算法步骤}

\subsection{探索(exploration)}
\subsubsection{每一只蚂蚁的每一步探索}
每一只蚂蚁对以其所在点为起点的边以信息素的量为概率随机选择一条边继续前进,若终点在当前循环中访问过则不访问,若随机5次均为访问过的节点那么蚂蚁这一回合就不移动,若移动则需在循环结束后更新信息素的量。
\vspace{0.6cm}
\begin{breakablealgorithm}
    \caption{每一步探索}
    \begin{algorithmic}[1]
        \Require $a$蚂蚁编号,$i$当前所在顶点编号
        \Function{Move}{$a,i$}
            \State $nAttemps\gets 0$
            \While{$nAttemps<5$}
                \State 根据信息素的量随机选取一条边$(i,j)$
                \If{顶点$j$未访问过}
                    \State 将边$(i,j)$标记为需要进行信息素更新
                    \State 将蚂蚁$a$移动至顶点$j$
                    \State 将顶点$j$标记为已被访问
                    \State break
                \Else
                    \State $nAttempts \gets nAttempts+1$
                \EndIf
            \EndWhile
        \EndFunction
    \end{algorithmic}
\end{breakablealgorithm}

\subsubsection{信息素更新}
信息素的更新不仅有自然衰减,而且也有一定的区间(上下界)
\vspace{0.6cm}
\begin{breakablealgorithm}
    \caption{信息素更新}
    \begin{algorithmic}[1]
        \Require 信息素更新队列
        \Ensure 新的信息素含量
        \Function{Update}{}
            \State $maxP \gets 1000((M-m)+(M-m)/3)$
            \State $minP \gets (M-m)/3$
            \While{信息素更新队列不为空}
                \State $P(i,j)\gets (1-\eta)P(i,j)+u(i,j)IP(i,j)$
                \If{$P(i,j)>maxP$}
                    \State $P(i,j)\gets maxP-IP(i,j)$
                \Else
                    \If{$P(i,j)<minP$}
                        \State $P(i,j)\gets minP+IP(i,j)$
                    \EndIf
                \EndIf
            \EndWhile
        \EndFunction
    \end{algorithmic}
\end{breakablealgorithm}

\subsection{建立(construction)}
本质为每次选取最优的N条边执行Kruskal算法,执行多次直到存在满足条件的最小生成树
\vspace{0.6cm}
\begin{breakablealgorithm}
    \caption{生成树}
    \begin{algorithmic}[1]
        \Require 图$G=(V,E)$,信息素水平$\omega$,度约束$d$
        \Ensure 当前满足条件的最小生成树
        \Function{Construct Tree}{$G,\omega,d$}
            \State 将所有边按照信息素水平排序
            \State 取$C$为信息素水平前n高的边
            \State 将$C$中的边按边权从小到大排序
            \State 对$C$做Kruskal,若未找到生成树,将$C$中再加入$n$条边做最小生成树直到找到为止
            \State \Return {找到的最小生成树}
        \EndFunction
    \end{algorithmic}
\end{breakablealgorithm}

\subsection{信息素调节}

每次找到最小生成树后都需要对信息素进行增强及上下界控制
其调节满足一下公式\footnote[1]{$\gamma$为增强常量,初始值为1.5,随程序不断运行而增大}:
$$
P(i,j)=
\left\{\begin{matrix}
\gamma P(i,j)
\\ 
maxP-IP(i,j)\ P(i,j)>maxP
\\ 
minP+IP(i,j)\ P(i,j)<minP
\end{matrix}\right.
$$

若在100次循环后,仍有一些边保留在集合里,我们人为将其信息素调整至其原始的[0.1,0.3]倍,防止程序误入一些疑似最优值的陷阱

\subsection{边界}

1、2500步内一直无更优解
\par
2、运行超过100000步


\chapter{实验仿真}
\section{数据设置}
图G的顶点数为N,对于每个顶点有随机值K条邻边,每条边相邻的点及边权也通过随机生成,为保证连通性其中$K \geq \frac{N}{2}$

\subsection{数据规模}
N分别取$100,1000,10000,50000,100000$进行测试,每种长度生成10组数据,每组数据运行5次,取总平均运行时间来代表其运行时间,取解的最小值来代表其运行结果
\subsection{测试环境}
系统:    Windows 10 Pro 1803 17134.112 \par
处理器:  Intel(R) Core(TM) i5-7300HQ CPU @ 2.50GHz 2.50GHz \par
运行内存:8.00 GB \par
编译环境:TDM64-GCC 4.9.2
\section{实验结果\protect\footnote[1]{对于运行结果的比较,可以参见MST问题的三篇论文的总报告}}
\subsection{运行时间}

\begin{table}[H]
  \caption{运行时间}
  \centering
  \begin{tabular}{p{2.6cm}<{\centering}p{1.3cm}<{\centering}p{1.3cm}<{\centering}p{1.3cm}<{\centering}p{1.3cm}<{\centering}p{1.3cm}<{\centering}p{1.3cm}<{\centering}}
  \toprule
   数据规模  & 50 & 100 & 200 & 300 & 400 & 500\\
  \midrule
   AB-DCST & 1.17s & 2.58s & 5.01s & 8.33s & 12.75s &15.17s\\
  \bottomrule
  \end{tabular} 
    
\end{table}
\chapter{结论和总结}
\section{对本算法的总结与思考}
由桶的定义可知,本文提出的算法仅适用于于整数型边权值数据的图,并且
由表\ref{tab22},表\ref{tab23},表\ref{tab24}可得在C较小的情况下1-level bucket较k-level bucket时间复杂度更小,在C较大时k-level bucket才有优化效果。
此种算法将所有桶建立成一个堆,从而快速寻找最值,是我了解到了优先队列的一种新的用法。
\par
通过研究本文算法,使我对如何对一个算法进行优化,优化的方向,以及如何结合不同的数据结构有了更深的理解。我认为,许多算法的改进都是建立在前面已有的算法知识的组合之上的,关键就在于如何确定优化的目标,确定目标后寻找适合的替代算法,还要注意不同算法之间结合的方式。

\backmatter   % 正文后无编号部分,
\chapter{致谢}
感谢陈松灿教授在算法研讨课上的讲授,这让我更深的理解了算法优化的一些知识与理论,给予了我极大的启发。
此外还要感谢赵子渊同学和张天凡同学在我阅读论文时提供的无私帮助,在他们的帮助下我才得以顺利完成此篇文章。 
\bibliographystyle{nuaathesis}   % 参考文献的样式
\bibliography{bib/Cite}   % 参考文献



\end{document}
