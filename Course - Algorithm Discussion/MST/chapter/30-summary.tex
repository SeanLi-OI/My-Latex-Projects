\chapter{结论和总结}
\section{对本算法的总结与思考}
对于启发式算法来说,虽然调参是很重要的一环,但主体还是如何设计一个合理的算法,如果算法不合理,通过调参的出来较优的解也不具有普适性。
启发式算法对具体数据依赖性很大,很不稳定,有些数据提前触发了结束边界可以在运行时间远小于理论值的情况下得出一个较优的解。
\par
在实践过程中,可以发现此篇论文提出的算法的大量时间运用在得出信息素水平后的生成树建立是通过Kruskal算法解决的。这也表明哦们可以通过启发式算法为传统算法提供指导信息,以达到优化传统算法的目的
\par
同时我也参考了L Wang\cite{Wang2003A}的遗传算法与启发式搜索结合解决DC-MST问题的算法,也为本文提出的算法中生成树部分提出了一个新的思路。