\chapter{实验仿真}
\section{数据设置}
图G的顶点数为N,对于每个顶点有随机值K条邻边,每条边相邻的点及边权也通过随机生成,为保证连通性其中$K \geq \frac{N}{2}$

\subsection{数据规模}
N分别取$100,1000,10000,50000,100000$进行测试,每种长度生成10组数据,每组数据运行5次,取总平均运行时间来代表其运行时间,取解的最小值来代表其运行结果
\subsection{测试环境}
系统:    Windows 10 Pro 1803 17134.112 \par
处理器:  Intel(R) Core(TM) i5-7300HQ CPU @ 2.50GHz 2.50GHz \par
运行内存:8.00 GB \par
编译环境:TDM64-GCC 4.9.2
\section{实验结果\protect\footnote[1]{对于运行结果的比较,可以参见MST问题的三篇论文的总报告}}
\subsection{运行时间}

\begin{table}[H]
  \caption{运行时间}
  \centering
  \begin{tabular}{p{2.6cm}<{\centering}p{1.3cm}<{\centering}p{1.3cm}<{\centering}p{1.3cm}<{\centering}p{1.3cm}<{\centering}p{1.3cm}<{\centering}p{1.3cm}<{\centering}}
  \toprule
   数据规模  & 50 & 100 & 200 & 300 & 400 & 500\\
  \midrule
   AB-DCST & 1.17s & 2.58s & 5.01s & 8.33s & 12.75s &15.17s\\
  \bottomrule
  \end{tabular} 
    
\end{table}